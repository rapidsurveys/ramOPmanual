\documentclass[12pt,]{book}
\usepackage{lmodern}
\usepackage{amssymb,amsmath}
\usepackage{ifxetex,ifluatex}
\usepackage{fixltx2e} % provides \textsubscript
\ifnum 0\ifxetex 1\fi\ifluatex 1\fi=0 % if pdftex
  \usepackage[T1]{fontenc}
  \usepackage[utf8]{inputenc}
\else % if luatex or xelatex
  \ifxetex
    \usepackage{mathspec}
  \else
    \usepackage{fontspec}
  \fi
  \defaultfontfeatures{Ligatures=TeX,Scale=MatchLowercase}
\fi
% use upquote if available, for straight quotes in verbatim environments
\IfFileExists{upquote.sty}{\usepackage{upquote}}{}
% use microtype if available
\IfFileExists{microtype.sty}{%
\usepackage{microtype}
\UseMicrotypeSet[protrusion]{basicmath} % disable protrusion for tt fonts
}{}
\usepackage[margin=1in]{geometry}
\usepackage{hyperref}
\hypersetup{unicode=true,
            pdftitle={A Guide to Implementing Nutrition and Food Security Surveys},
            pdfborder={0 0 0},
            breaklinks=true}
\urlstyle{same}  % don't use monospace font for urls
\usepackage{natbib}
\bibliographystyle{apalike}
\usepackage{color}
\usepackage{fancyvrb}
\newcommand{\VerbBar}{|}
\newcommand{\VERB}{\Verb[commandchars=\\\{\}]}
\DefineVerbatimEnvironment{Highlighting}{Verbatim}{commandchars=\\\{\}}
% Add ',fontsize=\small' for more characters per line
\usepackage{framed}
\definecolor{shadecolor}{RGB}{248,248,248}
\newenvironment{Shaded}{\begin{snugshade}}{\end{snugshade}}
\newcommand{\KeywordTok}[1]{\textcolor[rgb]{0.13,0.29,0.53}{\textbf{#1}}}
\newcommand{\DataTypeTok}[1]{\textcolor[rgb]{0.13,0.29,0.53}{#1}}
\newcommand{\DecValTok}[1]{\textcolor[rgb]{0.00,0.00,0.81}{#1}}
\newcommand{\BaseNTok}[1]{\textcolor[rgb]{0.00,0.00,0.81}{#1}}
\newcommand{\FloatTok}[1]{\textcolor[rgb]{0.00,0.00,0.81}{#1}}
\newcommand{\ConstantTok}[1]{\textcolor[rgb]{0.00,0.00,0.00}{#1}}
\newcommand{\CharTok}[1]{\textcolor[rgb]{0.31,0.60,0.02}{#1}}
\newcommand{\SpecialCharTok}[1]{\textcolor[rgb]{0.00,0.00,0.00}{#1}}
\newcommand{\StringTok}[1]{\textcolor[rgb]{0.31,0.60,0.02}{#1}}
\newcommand{\VerbatimStringTok}[1]{\textcolor[rgb]{0.31,0.60,0.02}{#1}}
\newcommand{\SpecialStringTok}[1]{\textcolor[rgb]{0.31,0.60,0.02}{#1}}
\newcommand{\ImportTok}[1]{#1}
\newcommand{\CommentTok}[1]{\textcolor[rgb]{0.56,0.35,0.01}{\textit{#1}}}
\newcommand{\DocumentationTok}[1]{\textcolor[rgb]{0.56,0.35,0.01}{\textbf{\textit{#1}}}}
\newcommand{\AnnotationTok}[1]{\textcolor[rgb]{0.56,0.35,0.01}{\textbf{\textit{#1}}}}
\newcommand{\CommentVarTok}[1]{\textcolor[rgb]{0.56,0.35,0.01}{\textbf{\textit{#1}}}}
\newcommand{\OtherTok}[1]{\textcolor[rgb]{0.56,0.35,0.01}{#1}}
\newcommand{\FunctionTok}[1]{\textcolor[rgb]{0.00,0.00,0.00}{#1}}
\newcommand{\VariableTok}[1]{\textcolor[rgb]{0.00,0.00,0.00}{#1}}
\newcommand{\ControlFlowTok}[1]{\textcolor[rgb]{0.13,0.29,0.53}{\textbf{#1}}}
\newcommand{\OperatorTok}[1]{\textcolor[rgb]{0.81,0.36,0.00}{\textbf{#1}}}
\newcommand{\BuiltInTok}[1]{#1}
\newcommand{\ExtensionTok}[1]{#1}
\newcommand{\PreprocessorTok}[1]{\textcolor[rgb]{0.56,0.35,0.01}{\textit{#1}}}
\newcommand{\AttributeTok}[1]{\textcolor[rgb]{0.77,0.63,0.00}{#1}}
\newcommand{\RegionMarkerTok}[1]{#1}
\newcommand{\InformationTok}[1]{\textcolor[rgb]{0.56,0.35,0.01}{\textbf{\textit{#1}}}}
\newcommand{\WarningTok}[1]{\textcolor[rgb]{0.56,0.35,0.01}{\textbf{\textit{#1}}}}
\newcommand{\AlertTok}[1]{\textcolor[rgb]{0.94,0.16,0.16}{#1}}
\newcommand{\ErrorTok}[1]{\textcolor[rgb]{0.64,0.00,0.00}{\textbf{#1}}}
\newcommand{\NormalTok}[1]{#1}
\usepackage{longtable,booktabs}
\usepackage{graphicx,grffile}
\makeatletter
\def\maxwidth{\ifdim\Gin@nat@width>\linewidth\linewidth\else\Gin@nat@width\fi}
\def\maxheight{\ifdim\Gin@nat@height>\textheight\textheight\else\Gin@nat@height\fi}
\makeatother
% Scale images if necessary, so that they will not overflow the page
% margins by default, and it is still possible to overwrite the defaults
% using explicit options in \includegraphics[width, height, ...]{}
\setkeys{Gin}{width=\maxwidth,height=\maxheight,keepaspectratio}
\IfFileExists{parskip.sty}{%
\usepackage{parskip}
}{% else
\setlength{\parindent}{0pt}
\setlength{\parskip}{6pt plus 2pt minus 1pt}
}
\setlength{\emergencystretch}{3em}  % prevent overfull lines
\providecommand{\tightlist}{%
  \setlength{\itemsep}{0pt}\setlength{\parskip}{0pt}}
\setcounter{secnumdepth}{5}
% Redefines (sub)paragraphs to behave more like sections
\ifx\paragraph\undefined\else
\let\oldparagraph\paragraph
\renewcommand{\paragraph}[1]{\oldparagraph{#1}\mbox{}}
\fi
\ifx\subparagraph\undefined\else
\let\oldsubparagraph\subparagraph
\renewcommand{\subparagraph}[1]{\oldsubparagraph{#1}\mbox{}}
\fi

%%% Use protect on footnotes to avoid problems with footnotes in titles
\let\rmarkdownfootnote\footnote%
\def\footnote{\protect\rmarkdownfootnote}

%%% Change title format to be more compact
\usepackage{titling}

% Create subtitle command for use in maketitle
\newcommand{\subtitle}[1]{
  \posttitle{
    \begin{center}\large#1\end{center}
    }
}

\setlength{\droptitle}{-2em}
  \title{A Guide to Implementing Nutrition and Food Security Surveys}
  \pretitle{\vspace{\droptitle}\centering\huge}
  \posttitle{\par}
  \author{}
  \preauthor{}\postauthor{}
  \predate{\centering\large\emph}
  \postdate{\par}
  \date{05/03/2018}

\usepackage{booktabs}

\usepackage{amsthm}
\newtheorem{theorem}{Theorem}[chapter]
\newtheorem{lemma}{Lemma}[chapter]
\theoremstyle{definition}
\newtheorem{definition}{Definition}[chapter]
\newtheorem{corollary}{Corollary}[chapter]
\newtheorem{proposition}{Proposition}[chapter]
\theoremstyle{definition}
\newtheorem{example}{Example}[chapter]
\theoremstyle{definition}
\newtheorem{exercise}{Exercise}[chapter]
\theoremstyle{remark}
\newtheorem*{remark}{Remark}
\newtheorem*{solution}{Solution}
\begin{document}
\maketitle

{
\setcounter{tocdepth}{1}
\tableofcontents
}
\hypertarget{introduction}{%
\chapter{Introduction}\label{introduction}}

Older people (generally defined as people aged sixty years and older)
are a vulnerable group for malnutrition in humanitarian and
developmental contexts. Due to their age they have specific nutritional
needs, such as easily digestible and palatable food adapted to those
with chewing problems, which is dense in nutrients. In famine and
displacement situations where populations are dependent on food
distributions, older people often find the general ration inappropriate
to their tastes and needs, have difficulties accessing the
distributions, or have difficulties transporting rations home. As a
result, older people can become malnourished and in need of specifically
targeted food interventions. In times of drought or food scarcity, older
people tend to reduce their food intake in order to share or give up
their ration to younger members of their families. They are then at risk
of malnutrition.

Despite these potential vulnerabilities in humanitarian situations,
older people are rarely identified as a group in need of specific
nutritional or food assistance. Surveys and assessments almost always
focus on children, and sometimes on pregnant and lactating women.
Humanitarian workers argue that assessing the nutritional status and
needs of older people is both costly and complicated. As a consequence,
the nutritional status and needs of older people in crisis go
unidentified and unaddressed.

HelpAge International, VALID International, and Brixton Health, with
financial assistance from the Humanitarian Innovation Fund (HIF), have
developed a Rapid Assessment Method for Older People (RAM-OP) that
provides accurate and reliable estimates of the needs of older people.
The method uses simple procedures, in a short time frame (i.e.~about two
weeks including training, data collection, data entry, and data
analysis), and at considerably lower cost than other methods. The RAM-OP
method is based on the following principles:

\begin{itemize}
\item
  Use of a familiar ``household survey'' design employing a two-stage
  cluster sample design optimised to allow the use of a small primary
  sample ( m ≥ 16 clusters) and a small overall ( n ≥ 192) sample.
\item
  Assessment of multiple dimensions of need in older people (including
  prevalence of global, moderate and severe acute malnutrition) using,
  whenever possible, standard and well-tested indicators and question
  sets.
\item
  Data analysis performed using modern computer-intensive methods to
  allow estimates of indicator levels to be made with useful precision
  using a small sample size.
\end{itemize}

The following tools are currently available under the General Public
Licence / Free Documentation License, meaning that you are free to copy
and adapt these tools:

\begin{itemize}
\item
  an English language manual / guidebook
\item
  a questionnaire (available in English and French)
\item
  data entry and data checking software (available in English and
  French)
\item
  data analysis software.
\end{itemize}

We believe that the availability of a rapid, low-cost, and user-friendly
method will encourage governments, UN agencies, as well as international
and local non-governmental organisations to actively assess the
situation of older people in humanitarian contexts, and implement,
monitor, and evaluate relevant and timely responses to address their
needs.

\hypertarget{sampling}{%
\chapter{The RAM-OP sample}\label{sampling}}

RAM-OP uses a two-stage sample:

\textbf{First stage sample:} A sample of communities (e.g.~villages or
city-blocks) in the survey area is taken. A sampled community is also
called a primary sampling unit (PSU).

\textbf{Second stage sample:} Domestic dwellings are sampled from within
the communities selected in the first stage sample. All eligible
individuals in the sampled dwelling are included in the sample.

\hypertarget{the-first-stage-sample}{%
\section{The first-stage sample}\label{the-first-stage-sample}}

The first stage sample is a systematic spatial sample. Two methods can
be used and both methods take the sample from all parts of the survey
area:

\textbf{List-based method:} Communities to be sampled are selected
systematically from a complete list of communities in the survey area.
This list of communities is sorted by one or more non-overlapping
spatial factors such as district and subdistricts within districts:

\hypertarget{indicators}{%
\chapter{Indicators}\label{indicators}}

\hypertarget{the-ram-op-questionnaire}{%
\chapter{The RAM-OP questionnaire}\label{the-ram-op-questionnaire}}

\hypertarget{datasets}{%
\chapter{Datasets}\label{datasets}}

\hypertarget{practical}{%
\chapter{Practical Fieldwork}\label{practical}}

\hypertarget{software}{%
\chapter{RAM-OP Software}\label{software}}

\hypertarget{data-entry}{%
\section{Data entry}\label{data-entry}}

\hypertarget{data-analysis}{%
\section{Data analysis}\label{data-analysis}}

This manual covers analysing your data using the \textbf{RAnalyticFlow}
workflow. An \textbf{RAnalyticFlow} workflow may be thought of as an
\emph{``app''} that makes it easy to analyse your survey data.

To use the \textbf{RAnalyticFlow} workflow you must install:

\begin{itemize}
\item
  \textbf{The \emph{R Language for Data-Analysis and Graphics} (R)} :
  This is the \emph{``engine''} which does all the work of analysing
  your data. You can get the R installation program from:
  \url{http://cran.r-project.org}. Following are links to download
  operating sofware-specific versions of R:

  \begin{itemize}
  \item
    \href{https://cran.r-project.org/bin/linux/}{Download R for Linux}
  \item
    \href{https://cran.r-project.org/bin/macosx/}{Download R for (Mac)
    OS X}
  \item
    \href{https://cran.r-project.org/bin/windows/}{Download R for
    Windowx}
  \end{itemize}
\item
  \textbf{R packages} (libraries of functions needed to work with the
  \textbf{RAnalyticFlow} workflow) : You can install these from within
  \textbf{R} using the Package Installer function within R. The
  libraries needed are:
\end{itemize}

\begin{longtable}[]{@{}ll@{}}
\toprule
\begin{minipage}[b]{0.25\columnwidth}\raggedright
\textbf{Package}\strut
\end{minipage} & \begin{minipage}[b]{0.69\columnwidth}\raggedright
\textbf{Comments}\strut
\end{minipage}\tabularnewline
\midrule
\endhead
\begin{minipage}[t]{0.25\columnwidth}\raggedright
\textbf{rJava}\strut
\end{minipage} & \begin{minipage}[t]{0.69\columnwidth}\raggedright
Required: Used by \textbf{RAnalyticFlow}\strut
\end{minipage}\tabularnewline
\begin{minipage}[t]{0.25\columnwidth}\raggedright
\textbf{JavaGD}\strut
\end{minipage} & \begin{minipage}[t]{0.69\columnwidth}\raggedright
Required: Used by \textbf{RAnalyticFlow}\strut
\end{minipage}\tabularnewline
\begin{minipage}[t]{0.25\columnwidth}\raggedright
\textbf{codetools}\strut
\end{minipage} & \begin{minipage}[t]{0.69\columnwidth}\raggedright
Required: Used by \textbf{RAnalyticFlow}\strut
\end{minipage}\tabularnewline
\begin{minipage}[t]{0.25\columnwidth}\raggedright
\textbf{foreign}\strut
\end{minipage} & \begin{minipage}[t]{0.69\columnwidth}\raggedright
Required: Opens \textbf{EpiData} (REC) files\strut
\end{minipage}\tabularnewline
\begin{minipage}[t]{0.25\columnwidth}\raggedright
\textbf{car}\strut
\end{minipage} & \begin{minipage}[t]{0.69\columnwidth}\raggedright
Required: Used for PROBIT estimator\strut
\end{minipage}\tabularnewline
\begin{minipage}[t]{0.25\columnwidth}\raggedright
\textbf{ggplot2}\strut
\end{minipage} & \begin{minipage}[t]{0.69\columnwidth}\raggedright
Desirable: Provides many plotting functions\strut
\end{minipage}\tabularnewline
\begin{minipage}[t]{0.25\columnwidth}\raggedright
\textbf{data.table}\strut
\end{minipage} & \begin{minipage}[t]{0.69\columnwidth}\raggedright
Desirable: Speeds up working with large dataset\strut
\end{minipage}\tabularnewline
\bottomrule
\end{longtable}

The Package Installer function can be called in R using the following
command:

\begin{Shaded}
\begin{Highlighting}[]
\KeywordTok{install.packages}\NormalTok{(}\KeywordTok{c}\NormalTok{(}\StringTok{"rJava"}\NormalTok{, }\StringTok{"JavaGD"}\NormalTok{, }\StringTok{"codetools"}\NormalTok{, }
                   \StringTok{"foreign"}\NormalTok{, }\StringTok{"car"}\NormalTok{, }\StringTok{"ggplot2"}\NormalTok{, }\StringTok{"data.table"}\NormalTok{), }
                   \DataTypeTok{repos =} \StringTok{"https://cloud.r-project.org/"}\NormalTok{)}
\end{Highlighting}
\end{Shaded}

The \texttt{repos} argument in the R command above specifies the CRAN
mirror from which you to download the package/s you want to install.
Here we specify the cloud-based mirror for CRAN provided by RStudio. If
unspecified, the installiation process will prompt you to select a
mirror from which to download packages from. If you already know the URL
of the CRAN mirror you want to use, specify this in the \texttt{repos}
argument.

Note that \textbf{RAnalyticFlow} may require you to have \textbf{Java}
installed. Check the instructions on the \textbf{RAnalyticFlow}
\href{http://r.analyticflow.com/en/download/}{download page} and on this
\href{http://download.ef-prime.com/ranalyticflow/3.1.5/readme.html}{starter
guide}.

All of this software is open source and free to download, copy, and use.
It will run on Windows, Mac OS X, and Linux (and other UNIX-like)
operating systems. Your ICT department should be able to help you with
installing this software.

In addition you will also need a copy of the \textbf{RAnalyticFlow}
workflow and supporting files. These are available from:

\url{http://www.brixtonhealth.com/ramOP.rflow.zip}

You may need to extract the file from the ZIP archive before use if this
is not done automatically.

\begin{figure}

{\centering \includegraphics[width=9.75in]{figures/dirStructureRAF} 

}

\caption{Directory structure of RAM-OP RAnalyticFlow package}\label{fig:raf1}
\end{figure}

Before starting to analyse your data you should create a project
directory or project folder. This is just a normal folder or directory
that can be created using your usual file manager (e.g.~Windows
ExplorerTM in WindowsTM or the FinderTM in Macintosh OS-XTM). The
project directory or project folder should contain:

\begin{enumerate}
\def\labelenumi{\arabic{enumi}.}
\item
  Your PSU file (here we assume this file is called testPSU.csv but it
  could have any name). This file must be a comma-separated-value (CSV)
  file.
\item
  Your survey data file (here we assume this file is called testSVY.rec
  but it could have any name). This file can be an EpiData (REC) file or
  a comma-separated-value (CSV) file.
\item
  The language file (always called ramOP.language.csv). This file
  provides text that is used in reports and graphics. The purpose of
  this file is to make the data analysis software produce reports in any
  language. This file must be a comma-separated-value (CSV) file.
\item
  A copy of the file ramOP.rflow.
\end{enumerate}

When you have created the project directory or project folder with the
required files you can start RAnalyticFlow.

Note: The \texttt{testSVY.rec} and \texttt{testPSU.csv} files are
example data files and are distributed with the \textbf{RAnalyticFlow}
workflow. You can use these files to practice analysing data using
\textbf{RAnalyticFlow}, and as examples of RAM-OP survey data and PSU
files.

\begin{figure}

{\centering \includegraphics[width=9.62in]{figures/openProjectRAF} 

}

\caption{Creating an RAnalyticFlow project}\label{fig:raf2}
\end{figure}

Before you start work you will need to create a project for your survey:

\begin{enumerate}
\def\labelenumi{\arabic{enumi}.}
\item
  Click the \textbf{New Project\ldots{}} button
\item
  Give your project a useful (i.e.~descriptive and memorable) name. This
  might be a name that describes the survey. For example, if the survey
  was done in the Kereinik locality of West Darfur in December 2015 you
  might use the name \textbf{WD.Kereinik.Dec2015.RAMOP}
\item
  Give the location of your project directory or project folder. This is
  the directory or folder which contains your survey data file, your PSU
  file, the RAM-OP language file, and a copy of \textbf{RAMOP.rflow}
  (see previous page). The location of the project directory or project
  folder (labeled ``Path'' by the software) that \textbf{RAnalyticFlow}
  selects automatically will almost always be wrong. You need to specify
  this manually.
\item
  Click the \textbf{OK} button
\end{enumerate}

\begin{figure}

{\centering \includegraphics[width=10.12in]{figures/openWorkflowRAF} 

}

\caption{Open an RAnalyticFlow workflow}\label{fig:raf3}
\end{figure}

Double click the item named \textbf{ramOP.rflow} shown in the file
manager pane of the \textbf{RAnalyticFlow} window. This will open the
data-analysis workflow which will be shown in the workflow viewer /
editor window of the \textbf{RAnalyticFlow} window.

Once you have opened the workflow you need to initialise it (i.e.~load
libraries, useful analysis function, and initialise the workspace for a
new analysis):

\begin{figure}

{\centering \includegraphics[width=10.12in]{figures/runWorkflowRAF} 

}

\caption{Run an RAnalyticFlow workflow}\label{fig:raf4}
\end{figure}

Once this is done, you should:

\begin{enumerate}
\def\labelenumi{\arabic{enumi}.}
\item
  Retrieve your survey data. This can be in EpiDat (REC) format or CSV
  format. Select and run the appropriate \textbf{Survey Data} node and
  select the survey data file.
\item
  Retrieve the PSU date data. Select and run the \textbf{PSU Data} node
  and select your PSU file.
\item
  Produce the survey report and graphics. Select and run the
  \textbf{Report} node. This will take some time to complete because the
  analysis uses computer intensive techniques to make best use of the
  available data. The Report node/icon will have black lines around it
  has completed running the report.
\end{enumerate}

\hypertarget{conclusion}{%
\chapter{Conclusion}\label{conclusion}}

We live in an ageing world, where people aged 60 or over will be 2
billion or about 22\% of the world's population by 2050.

Currently, two in three people aged 60 years or older live in developing
countries. By 2050, nearly four in five older people will be living in
the developing world.

The changing demographics of ageing combined with the increasing number
of disasters will exert a disproportionate impact on the world's oldest
and poorest.

In this context, identifying the needs of older people as accurately as
possible is a necessity. More and more donors and UN agencies are now
willing to include older people in their programmes. Age markers, to
complement gender markers, will be disseminated very soon

RAM-OP is offering a fast, robust, reliable, tested and user-friendly
way of assessing the needs of older people. It can be used in
humanitarian situations as well as in development contexts. The modular
structure of RAM-OP allows for adaptations, making it exhaustive or
limited to essential indicators according to the immediate needs.

As more organisations start to use it, RAM-OP will evolve and improve.
New versions of RAM-OP can be created (for example, RAM-OP for refugee
or displaced people camps). We wish that a greater number of actors will
start using RAM-OP and make it their own.

\bibliography{book.bib}


\end{document}
