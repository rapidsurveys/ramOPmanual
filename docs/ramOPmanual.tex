\documentclass[12pt,]{book}
\usepackage{lmodern}
\usepackage{amssymb,amsmath}
\usepackage{ifxetex,ifluatex}
\usepackage{fixltx2e} % provides \textsubscript
\ifnum 0\ifxetex 1\fi\ifluatex 1\fi=0 % if pdftex
  \usepackage[T1]{fontenc}
  \usepackage[utf8]{inputenc}
\else % if luatex or xelatex
  \ifxetex
    \usepackage{mathspec}
  \else
    \usepackage{fontspec}
  \fi
  \defaultfontfeatures{Ligatures=TeX,Scale=MatchLowercase}
\fi
% use upquote if available, for straight quotes in verbatim environments
\IfFileExists{upquote.sty}{\usepackage{upquote}}{}
% use microtype if available
\IfFileExists{microtype.sty}{%
\usepackage{microtype}
\UseMicrotypeSet[protrusion]{basicmath} % disable protrusion for tt fonts
}{}
\usepackage[margin=1in]{geometry}
\usepackage{hyperref}
\PassOptionsToPackage{usenames,dvipsnames}{color} % color is loaded by hyperref
\hypersetup{unicode=true,
            pdftitle={Rapid Assessment Method for Older People (RAM-OP): The Manual},
            pdfauthor={Pascale Fritsch, Ernest Guevarra, Katja Siling, Mark Myatt},
            colorlinks=true,
            linkcolor=Maroon,
            citecolor=Blue,
            urlcolor=Blue,
            breaklinks=true}
\urlstyle{same}  % don't use monospace font for urls
\usepackage{natbib}
\bibliographystyle{apalike}
\usepackage{color}
\usepackage{fancyvrb}
\newcommand{\VerbBar}{|}
\newcommand{\VERB}{\Verb[commandchars=\\\{\}]}
\DefineVerbatimEnvironment{Highlighting}{Verbatim}{commandchars=\\\{\}}
% Add ',fontsize=\small' for more characters per line
\usepackage{framed}
\definecolor{shadecolor}{RGB}{248,248,248}
\newenvironment{Shaded}{\begin{snugshade}}{\end{snugshade}}
\newcommand{\KeywordTok}[1]{\textcolor[rgb]{0.13,0.29,0.53}{\textbf{#1}}}
\newcommand{\DataTypeTok}[1]{\textcolor[rgb]{0.13,0.29,0.53}{#1}}
\newcommand{\DecValTok}[1]{\textcolor[rgb]{0.00,0.00,0.81}{#1}}
\newcommand{\BaseNTok}[1]{\textcolor[rgb]{0.00,0.00,0.81}{#1}}
\newcommand{\FloatTok}[1]{\textcolor[rgb]{0.00,0.00,0.81}{#1}}
\newcommand{\ConstantTok}[1]{\textcolor[rgb]{0.00,0.00,0.00}{#1}}
\newcommand{\CharTok}[1]{\textcolor[rgb]{0.31,0.60,0.02}{#1}}
\newcommand{\SpecialCharTok}[1]{\textcolor[rgb]{0.00,0.00,0.00}{#1}}
\newcommand{\StringTok}[1]{\textcolor[rgb]{0.31,0.60,0.02}{#1}}
\newcommand{\VerbatimStringTok}[1]{\textcolor[rgb]{0.31,0.60,0.02}{#1}}
\newcommand{\SpecialStringTok}[1]{\textcolor[rgb]{0.31,0.60,0.02}{#1}}
\newcommand{\ImportTok}[1]{#1}
\newcommand{\CommentTok}[1]{\textcolor[rgb]{0.56,0.35,0.01}{\textit{#1}}}
\newcommand{\DocumentationTok}[1]{\textcolor[rgb]{0.56,0.35,0.01}{\textbf{\textit{#1}}}}
\newcommand{\AnnotationTok}[1]{\textcolor[rgb]{0.56,0.35,0.01}{\textbf{\textit{#1}}}}
\newcommand{\CommentVarTok}[1]{\textcolor[rgb]{0.56,0.35,0.01}{\textbf{\textit{#1}}}}
\newcommand{\OtherTok}[1]{\textcolor[rgb]{0.56,0.35,0.01}{#1}}
\newcommand{\FunctionTok}[1]{\textcolor[rgb]{0.00,0.00,0.00}{#1}}
\newcommand{\VariableTok}[1]{\textcolor[rgb]{0.00,0.00,0.00}{#1}}
\newcommand{\ControlFlowTok}[1]{\textcolor[rgb]{0.13,0.29,0.53}{\textbf{#1}}}
\newcommand{\OperatorTok}[1]{\textcolor[rgb]{0.81,0.36,0.00}{\textbf{#1}}}
\newcommand{\BuiltInTok}[1]{#1}
\newcommand{\ExtensionTok}[1]{#1}
\newcommand{\PreprocessorTok}[1]{\textcolor[rgb]{0.56,0.35,0.01}{\textit{#1}}}
\newcommand{\AttributeTok}[1]{\textcolor[rgb]{0.77,0.63,0.00}{#1}}
\newcommand{\RegionMarkerTok}[1]{#1}
\newcommand{\InformationTok}[1]{\textcolor[rgb]{0.56,0.35,0.01}{\textbf{\textit{#1}}}}
\newcommand{\WarningTok}[1]{\textcolor[rgb]{0.56,0.35,0.01}{\textbf{\textit{#1}}}}
\newcommand{\AlertTok}[1]{\textcolor[rgb]{0.94,0.16,0.16}{#1}}
\newcommand{\ErrorTok}[1]{\textcolor[rgb]{0.64,0.00,0.00}{\textbf{#1}}}
\newcommand{\NormalTok}[1]{#1}
\usepackage{longtable,booktabs}
\usepackage{graphicx,grffile}
\makeatletter
\def\maxwidth{\ifdim\Gin@nat@width>\linewidth\linewidth\else\Gin@nat@width\fi}
\def\maxheight{\ifdim\Gin@nat@height>\textheight\textheight\else\Gin@nat@height\fi}
\makeatother
% Scale images if necessary, so that they will not overflow the page
% margins by default, and it is still possible to overwrite the defaults
% using explicit options in \includegraphics[width, height, ...]{}
\setkeys{Gin}{width=\maxwidth,height=\maxheight,keepaspectratio}
\IfFileExists{parskip.sty}{%
\usepackage{parskip}
}{% else
\setlength{\parindent}{0pt}
\setlength{\parskip}{6pt plus 2pt minus 1pt}
}
\setlength{\emergencystretch}{3em}  % prevent overfull lines
\providecommand{\tightlist}{%
  \setlength{\itemsep}{0pt}\setlength{\parskip}{0pt}}
\setcounter{secnumdepth}{5}
% Redefines (sub)paragraphs to behave more like sections
\ifx\paragraph\undefined\else
\let\oldparagraph\paragraph
\renewcommand{\paragraph}[1]{\oldparagraph{#1}\mbox{}}
\fi
\ifx\subparagraph\undefined\else
\let\oldsubparagraph\subparagraph
\renewcommand{\subparagraph}[1]{\oldsubparagraph{#1}\mbox{}}
\fi

%%% Use protect on footnotes to avoid problems with footnotes in titles
\let\rmarkdownfootnote\footnote%
\def\footnote{\protect\rmarkdownfootnote}

%%% Change title format to be more compact
\usepackage{titling}

% Create subtitle command for use in maketitle
\newcommand{\subtitle}[1]{
  \posttitle{
    \begin{center}\large#1\end{center}
    }
}

\setlength{\droptitle}{-2em}
  \title{Rapid Assessment Method for Older People (RAM-OP): The Manual}
  \pretitle{\vspace{\droptitle}\centering\huge}
  \posttitle{\par}
  \author{Pascale Fritsch, Ernest Guevarra, Katja Siling, Mark Myatt}
  \preauthor{\centering\large\emph}
  \postauthor{\par}
  \predate{\centering\large\emph}
  \postdate{\par}
  \date{21/12/2015}

\usepackage{booktabs}
\usepackage{color}
\usepackage{tcolorbox}
\usepackage{float}
\graphicspath{ {images/} }

\newenvironment{rmdremind}
  {\begin{tcolorbox}[width=\textwidth, 
                     colback = {white}, 
                     title = {\textbf{Remember}}, 
                     colbacktitle = lightgray,
                     coltitle = black]
  \begin{includegraphics}[scale = 1]{remind.png}
  \begin{itemize}}
  {\end{itemize}
  \end{includegraphics}
  \end{tcolorbox}}

\newenvironment{rmdnote}
  {\begin{tcolorbox}[width=\textwidth, 
                     colback = {white}, 
                     title = {\textbf{Note}}, 
                     colbacktitle = lightgray,
                     coltitle = black]
  \begin{includegraphics}[scale = 1]{pencil.png}}
  {\end{includegraphics}
  \end{tcolorbox}}
  
\newenvironment{rmdexercise}
  {\begin{tcolorbox}[width=\textwidth, 
                     colback = {white}, 
                     title = {\textbf{Exercise}}, 
                     colbacktitle = lightgray,
                     coltitle = black]
  \begin{includegraphics}[scale = 1]{exercise.png}}
  {\end{includegraphics}
  \end{tcolorbox}}
  
\newenvironment{rmdinfo}
  {\begin{tcolorbox}[width=\textwidth, 
                     colback = {white}, 
                     title = {\textbf{Info}}, 
                     colbacktitle = lightgray,
                     coltitle = black]
  \begin{includegraphics}[scale = 1]{info.png}}
  {\end{includegraphics}
  \end{tcolorbox}}  
  
\newenvironment{rmdwarning}
  {\begin{tcolorbox}[width=\textwidth, 
                     colback = {white}, 
                     title = {\textbf{Warning}}, 
                     colbacktitle = lightgray,
                     coltitle = black]
  \begin{includegraphics}[scale = 1]{warning.png}}
  {\end{includegraphics}
  \end{tcolorbox}}

\newenvironment{rmddownload}
  {\begin{tcolorbox}[width=\textwidth, 
                     colback = {white}, 
                     title = {\textbf{Download}}, 
                     colbacktitle = lightgray,
                     coltitle = black]
  \begin{includegraphics}[scale = 1]{download.png}}
  {\end{includegraphics}
  \end{tcolorbox}}

\usepackage{amsthm}
\newtheorem{theorem}{Theorem}[chapter]
\newtheorem{lemma}{Lemma}[chapter]
\theoremstyle{definition}
\newtheorem{definition}{Definition}[chapter]
\newtheorem{corollary}{Corollary}[chapter]
\newtheorem{proposition}{Proposition}[chapter]
\theoremstyle{definition}
\newtheorem{example}{Example}[chapter]
\theoremstyle{definition}
\newtheorem{exercise}{Exercise}[chapter]
\theoremstyle{remark}
\newtheorem*{remark}{Remark}
\newtheorem*{solution}{Solution}
\begin{document}
\maketitle

{
\hypersetup{linkcolor=black}
\setcounter{tocdepth}{1}
\tableofcontents
}
\hypertarget{the-ram-op-manual}{%
\chapter*{The RAM-OP Manual}\label{the-ram-op-manual}}
\addcontentsline{toc}{chapter}{The RAM-OP Manual}

\includegraphics{figures/coverImage.jpg}

\hypertarget{introduction}{%
\chapter*{Introduction}\label{introduction}}
\addcontentsline{toc}{chapter}{Introduction}

Older people (generally defined as people aged sixty years and older)
are a vulnerable group for malnutrition in humanitarian and
developmental contexts. Due to their age they have specific nutritional
needs, such as easily digestible and palatable food adapted to those
with chewing problems, which is dense in nutrients. In famine and
displacement situations where populations are dependent on food
distributions, older people often find the general ration inappropriate
to their tastes and needs, have difficulties accessing the
distributions, or have difficulties transporting rations home. As a
result, older people can become malnourished and in need of specifically
targeted food interventions. In times of drought or food scarcity, older
people tend to reduce their food intake in order to share or give up
their ration to younger members of their families. They are then at risk
of malnutrition.

Despite these potential vulnerabilities in humanitarian situations,
older people are rarely identified as a group in need of specific
nutritional or food assistance. Surveys and assessments almost always
focus on children, and sometimes on pregnant and lactating women.
Humanitarian workers argue that assessing the nutritional status and
needs of older people is both costly and complicated. As a consequence,
the nutritional status and needs of older people in crisis go
unidentified and unaddressed.

HelpAge International, VALID International, and Brixton Health, with
financial assistance from the Humanitarian Innovation Fund (HIF), have
developed a Rapid Assessment Method for Older People (RAM-OP) that
provides accurate and reliable estimates of the needs of older people.
The method uses simple procedures, in a short time frame (i.e.~about two
weeks including training, data collection, data entry, and data
analysis), and at considerably lower cost than other methods. The RAM-OP
method is based on the following principles:

\begin{itemize}
\item
  Use of a familiar ``household survey'' design employing a two-stage
  cluster sample design optimised to allow the use of a small primary
  sample ( m ≥ 16 clusters) and a small overall ( n ≥ 192) sample.
\item
  Assessment of multiple dimensions of need in older people (including
  prevalence of global, moderate and severe acute malnutrition) using,
  whenever possible, standard and well-tested indicators and question
  sets.
\item
  Data analysis performed using modern computer-intensive methods to
  allow estimates of indicator levels to be made with useful precision
  using a small sample size.
\end{itemize}

The following tools are currently available under the General Public
Licence / Free Documentation License, meaning that you are free to copy
and adapt these tools:

\begin{itemize}
\item
  an English language manual / guidebook
\item
  a questionnaire (available in English and French)
\item
  data entry and data checking software (available in English and
  French)
\item
  data analysis software.
\end{itemize}

We believe that the availability of a rapid, low-cost, and user-friendly
method will encourage governments, UN agencies, as well as international
and local non-governmental organisations to actively assess the
situation of older people in humanitarian contexts, and implement,
monitor, and evaluate relevant and timely responses to address their
needs.

\hypertarget{sampling}{%
\chapter{Sampling}\label{sampling}}

\hypertarget{the-ram-op-sample}{%
\section{The RAM-OP sample}\label{the-ram-op-sample}}

RAM-OP uses a two-stage sample:

\textbf{First stage sample:} A sample of communities (e.g.~villages or
city-blocks) in the survey area is taken. A sampled community is also
called a primary sampling unit (PSU).

\textbf{Second stage sample:} Domestic dwellings are sampled from within
the communities selected in the first stage sample. All eligible
individuals in the sampled dwelling are included in the sample.

\hypertarget{the-first-stage-sample}{%
\subsection{The first-stage sample}\label{the-first-stage-sample}}

The first stage sample is a systematic spatial sample. Two methods can
be used and both methods take the sample from all parts of the survey
area:

\begin{itemize}
\tightlist
\item
  \textbf{List-based method:} Communities to be sampled are selected
  systematically from a complete list of communities in the survey area.
  This list of communities is sorted by one or more non-overlapping
  spatial factors such as district and subdistricts within districts:
\end{itemize}

\begin{figure}[H]

{\centering \includegraphics{figures/listSample1} 

}

\caption{Communities listing by district and sub-district}\label{fig:sample1}
\end{figure}

\begin{itemize}
\tightlist
\item
  \textbf{Map-based method:} Communities to be sampled are selected from
  the centres of the squares of a grid drawn over a map. The map must be
  sufficiently well made and of sufficiently large scale to show the
  position of every community in the survey area. This type of sample is
  known as a centric systematic area sample and is often referred to as
  a CSAS sample.
\end{itemize}

\textbf{Note:} \emph{Population proportional sampling} (PPS) is
\textbf{not} used in RAM-OP surveys. Population estimates for all
communities are \textbf{not} required for sampling purposes. Population
estimates are required only for the selected communities. These are used
during data analysis in order to weight results by population size. If
this information is not available before the survey, it can be collected
during the survey.

\hypertarget{the-second-stage-sample}{%
\subsection{The second stage sample}\label{the-second-stage-sample}}

The second stage within-community sample uses a method called
map-segment-sample. This method takes the within-community sample from
all parts of a sampled community.

\hypertarget{implicit-stratification}{%
\section{Implicit stratification}\label{implicit-stratification}}

Both the first and second stage samples use a form of spatial
stratification:

\begin{itemize}
\item
  The list-based method's first stage systematic spatial sample
  stratifies the sample by non-overlapping spatial factor such as
  districts and subdistricts within districts.
\item
  The map-based (CSAS) method's first stage sample stratifies the sample
  by grid square.
\item
  The map-segment-sample second stage within-community sample stratifies
  the sample by parts of the community being sampled.
\item
  The first and second stage samples also ensure that a reasonably even
  spatial sample is taken from the entire survey area and from each of
  the sampled communities.
\end{itemize}

These sampling procedures provide \emph{implicit stratification} and
tend to spread the sample properly among important sub-groups of the
population such as rural / urban / peri-urban populations,
administrative areas, ethnic sub-populations, religious sub-populations,
and socio-economic groups. This often improves the precision of
estimates made from survey data.

The use of implicit stratification improves the efficiency of a
two-stage cluster sample and allows RAM-OP to use relatively small
sample sizes compared to other methods, such as SMART surveys. The use
of modern computer-intensive data analysis techniques also allows RAM-OP
to make better use of the available sample than is done in other
methods.

\hypertarget{ram-op-survey-sample-size}{%
\section{RAM-OP survey sample size}\label{ram-op-survey-sample-size}}

The following shorthand symbols will be used when describing sample
designs:

\begin{align*}
m &= \text{Number of primary sampling units (PSUs).} \\
n &= \text{Size of the sample of individuals or households from a PSU.} \\
n &= \text{May also mean the overall survey sample size (this meaning will be made clear in the text).} \\
N &= \text{Population}
\end{align*}

The overall sample size for a RAM-OP survey is about \(n = 192\)
individual subjects. You should aim to collect an overall sample of at
least \(n = 192\) individuals.

The RAM-OP sample is collected in two stages:

\begin{itemize}
\item
  The first stage sample uses a sample size of about \(m = 16\)
  communities (or PSUs).
\item
  The second stage sample uses a sample size of about \(n = 12\)
  eligible subjects sampled from each of the communities selected for
  inclusion in the first stage sample.
\end{itemize}

The overall sample size from \(m = 16\) communities and \(n = 12\)
eligible subjects is about:

\[ \text{overall sample size} ~ \approx ~ m ~ \times ~ n ~ \approx ~ 16 ~ \times ~ 12 ~ \approx ~ 192 \]

It is not recommended that fewer than \(m = 16\) communities are
sampled.

\hypertarget{ram-op-survey-sample-size-1}{%
\section{RAM-OP survey sample size}\label{ram-op-survey-sample-size-1}}

Sampling fewer than \(m = 16\) communities will tend to reduce the
precision with which estimates can be made. If you have the resources to
sample more than \(m = 16\) communities then you should do so. A sample
of \(m = 24\) communities and \(n = 8\) eligible subjects, for example,
will tend to yield estimates with better precision than a sample with
\(m = 16\) communities and \(n = 12\) eligible subjects.

Do not be tempted to increase the size of the within-community sample in
order to achieve an overall sample size of \(n = 192\) from fewer than
\(m = 16\) communities. Doing so will tend to reduce the precision with
which estimates are made. It may also be impossible to do this in many
settings.

Here, for example, is a \emph{population pyramid} for a typical
developing country:

\begin{figure}[H]

{\centering \includegraphics{figures/popPyramid1} 

}

\caption{Population pyramid for a typical developing country}\label{fig:sample2}
\end{figure}

If the average community population is \(N = 300\) then there will be
fewer than 15 people aged 60 years and older in about half of the
selected communities. This is because about half of the selected
communities are likely to have a population below the average
population.

\hypertarget{eligibility}{%
\section{Eligibility}\label{eligibility}}

Older people are usual defined as persons aged 60 years and older (UN
definition). This means your sample will usually be restricted to people
aged 60 years and older.

In some settings different eligibility criteria may apply. This will
likely be the case in settings with very high life-expectancies (usually
middle and high income countries) or very low life-expectancies (usually
low income countries and in emergencies).

In a setting of very high life-expectancy you may want to restrict
eligibility - to persons aged 65 years or older, for example. A local
definition of older people is likely to be available.

In a setting with very low life-expectancy, very few people are aged 60
years or older. For example:

\begin{figure}[H]

{\centering \includegraphics{figures/popPyramid2} 

}

\caption{Population pyramid for a setting with low life-expectancy}\label{fig:sample3}
\end{figure}

It is common in such setting for there to be a local definition of older
people. This will usually be ``persons aged 50 years or older'' or
``persons aged 55 years or older''.

\hypertarget{age-distribution-eligibility-criteria-and-sample-design}{%
\section{Age distribution, eligibility criteria, and sample
design}\label{age-distribution-eligibility-criteria-and-sample-design}}

The age distribution of the population and the survey eligibility
criteria will affect the sample design in terms of the number of
communities that you will need to sample (\(m\)) and the number of older
persons (\(n\)) that can be sampled from each community.

The overall sample size for a RAM-OP sample should be at least
\(n = 192\) usually collected as \(n = 12\) eligible subjects sampled
from \(m = 16\) communities. If older people make up a very small
proportion (i.e.~much less than 5\%) of the total population and / or
the average population of communities is small then you will usually
need to sample more than m = 16 communities in order to get about
\(n = 192\) older people in the overall sample. This is likely to occur
when there are fewer than 20 to 25 older people in a community of
average size.

You can calculate the number of older people that you would expect to be
living in a community of average size using the following formula:

\[ n_{\text{aged 60+ in an average village}} ~ = ~ \text{average village population}_{\text{all ages}} ~ \times ~ \frac{\text{percentage of population}_{\text{aged 60+}}}{100} \]

If this is below about 20 people then you should consider how you will
collect the required overall sample size. Three approaches may be used:

\begin{itemize}
\item
  \textbf{Relax the eligibility criteria:} You may decide to define
  older people as ``persons aged 50 years or older'' or ``persons aged
  55 years or older''. This may double the size of the eligible
  population and make the sample easier to collect. This approach is
  only reasonable if life-expectancy is low.
\item
  \textbf{Increase the number of communities that you plan to sample:}
  You may choose to collect your sample as \(n = 7\) eligible subjects
  sampled from \(m = 30\) communities giving an expected overall sample
  size of \(n = 210\). This would be a very good sample. The
  disadvantage of this approach is that survey costs increase with the
  number of communities that are sampled, because a lot of survey time
  and vehicle costs are spent on travelling to and from the selected
  communities.
\item
  \textbf{Take a ``top-up'' sample only when you need to:} The basic
  procedure when a selected community is small and likely to contain
  fewer than \(n = 12\) older people is to collect data on all older
  people in the selected community using a door-to-door census. If the
  within-community sample size is much smaller than the required one
  then a ``top-up'' sample is taken from the nearest neighbouring
  community using the map-segment-sample method (or a door-to-door
  census if this community is also small). The advantage of this
  approach is that travelling time and survey costs are better
  controlled.
\end{itemize}

If the proportion of older people is not very small and / or communities
are large then you should have no problems achieving the overall sample
size.

\hypertarget{practical-sampling}{%
\section{Practical sampling}\label{practical-sampling}}

\hypertarget{the-first-stage-sample---list-based-sampling}{%
\subsection{The first stage sample - list-based
sampling}\label{the-first-stage-sample---list-based-sampling}}

The first stage sample can be drawn from a list of all communities. The
list-based sample is a simple systematic sample taken from a complete
list of communities in the survey area sorted by one or more non-
overlapping spatial factors (such as administrative units or electoral
wards) in the survey area. \emph{Population proportional sampling} (PPS)
is not used since this would concentrate the sample in the larger
communities.

Below is a worked example of how a RAM-OP first stage, list-based sample
can be drawn from a survey area composed of 67 villages.

\textbf{Step 1:} Calculate the \emph{sampling interval} by dividing the
total villages in the survey area (67 villages) with the number of
villages to be drawn from the sample (16 villages).

\[ \text{Sampling Interval} ~ = ~ \left \lfloor ~ \frac{N_{villages}}{N_{sample}} ~ \right \rfloor ~ = ~ \left \lfloor ~ \frac{67}{16} ~ \right \rfloor ~ \approx ~ \left \lfloor ~ 4.19 ~ \right \rfloor ~ \approx ~ 4 \]

The \emph{sampling interval} needs to be a whole number. Remember to
\textbf{always} \emph{round down} when calculating the \emph{sampling
interval} to the nearest whole number.

\textbf{Step 2:} Choose a \emph{random starting point} between 1 and
\emph{sampling interval}. In this example, this would be a random number
\textbf{between 1 and 4}.

A random number can be selected through simple lottery (i.e., draw from
a lot of 4 numbered from 1 to 4). A standard spreadsheet software can
also be used to draw the random number using the \texttt{RANDBETWEEN}
function as follows:

\texttt{RANDBETWEEN(1,\ 4)}

\textbf{Step 3:} Using the \emph{random starting point} and the
\emph{sampling interval}, select the sampling villages from a list of
all villages organised/sorted by a \textbf{non-overalapping} spatial
factor such as district or sub-district.

\begin{figure}[H]

{\centering \includegraphics{figures/listSample2} 

}

\caption{Selection of sampling villages using lists}\label{fig:sample4}
\end{figure}

This procedure will sometimes select more than 16 communities.In this
example, seventeen villages (i.e.~at positions 2, 6, 10, 14, 18, 22, 26,
30, 34, 38, 42, 46, 50, 54, 58, 62, and 66 in the list) will be
selected.When this happens you should sample \textbf{all} of the
selected communities.

\hypertarget{the-first-stage-sample---map-based-sampling}{%
\subsection{The first stage sample - map-based
sampling}\label{the-first-stage-sample---map-based-sampling}}

An alternative approach to list-based sampling is to use map-based
sampling. The map-based (CSAS) sample selects communities from the
centre of squares of a grid drawn over a map. The map must be
sufficiently well made and of sufficiently large scale to show the
position of \textbf{all} communities in the survey area.

A square grid is drawn over the map. The size of the grid squares should
be small enough so that the number of squares covering the survey area
is the same as (or very similar to) the number of communities that you
plan to sample. You may need to experiment with different grid sizes to
achieve this. Figure \ref{fig:sample6} shows an example map and grid
with \(m = 16\) grid squares.

The sample is drawn by selecting the community that is located closest
to the centre of each grid square:

\begin{figure}[H]

{\centering \includegraphics{figures/mapSample1} 

}

\caption{Selection of sampling villages using maps}\label{fig:sample5}
\end{figure}

If two or more villages are located the same distance from the centre of
a grid square then a single village is picked at random, by tossing a
coin for example.

Figure \ref{fig:sample7} shows the sample selected by this process for
the area shown in Figure \ref{fig:sample6}.

\begin{figure}[H]

{\centering \includegraphics{figures/mapSample2} 

}

\caption{Drawing a square grid over the map}\label{fig:sample6}
\end{figure}

\begin{figure}[H]

{\centering \includegraphics{figures/mapSample3} 

}

\caption{Drawing the first-stage CSAS sample}\label{fig:sample7}
\end{figure}

Both the list-based and the map-based (CSAS) sampling methods spread the
sample of communities evenly across the entire survey area. Each
community has an equal chance of being included in the sample.
Population proportional sampling (PPS) is not used since this would
concentrate the sample in the larger communities.

The same method can be used when sampling in urban contexts. Figure
\ref{fig:sample8} shows a sample drawn from a list of census enumeration
areas sorted by administrative district. Figure \ref{fig:sample9} shows
a sample drawn using the map- based (CSAS) method. In both cases the
primary sampling units (PSUs) are census enumeration areas.

\begin{figure}[H]

{\centering \includegraphics{figures/mapSample4} 

}

\caption{Example of an urban sample (list-based)}\label{fig:sample8}
\end{figure}

\begin{figure}[H]

{\centering \includegraphics{figures/mapSample5} 

}

\caption{Example of an urban sample (map-based)}\label{fig:sample9}
\end{figure}

\textbf{Note:} In this example twenty-one (21) blocks have been
selected. It can be difficult to achieve exactly the number of blocks
that you need when using this type of sample. It is best to select more
rather than fewer blocks than you need Here we would take our sample as
\(n = 10\) individuals from \(m = 21\) blocks (overall \(n = 210\)).

\hypertarget{the-second-stage-within-community-sample}{%
\subsection{The second stage (within-community)
sample}\label{the-second-stage-within-community-sample}}

The second stage (within-community) sample uses a map-segment-sample
approach:

\textbf{Map:} Make a rough map of the community to be sampled. It is
helpful to think of communities as being made of ribbons (i.e.~lines of
dwellings located along roads, tracks, or rivers) and clusters of
dwellings.

Here is an example of a ribbon of dwellings:

\begin{figure}[H]

{\centering \includegraphics{figures/stage2sample1} 

}

\caption{Example of a ribbon of dwellings}\label{fig:sample10}
\end{figure}

Here is an example of a cluster of dwellings:

\begin{figure}[H]

{\centering \includegraphics{figures/stage2sample2} 

}

\caption{Example of a cluster of dwellings}\label{fig:sample11}
\end{figure}

\textbf{Segment:} Divide the community into ribbon and cluster segments
defined by the physical layout of the community being sampled.

\textbf{Sample:} Ribbons and clusters are sampled in different ways:

\begin{itemize}
\tightlist
\item
  \textbf{Ribbons} are sampled using \textbf{systematic sampling}.
\item
  \textbf{Clusters} are sampled using a \textbf{random walk} method.
\end{itemize}

\textbf{Note:} If a small community is selected that is likely to have
fewer than the required number of eligible persons then \textbf{all}
eligible persons in that community are sampled by moving door-to-door.

\hypertarget{mapping-the-community---single-and-multiple-clusters}{%
\subsection{Mapping the community - single and multiple
clusters}\label{mapping-the-community---single-and-multiple-clusters}}

Some communities consist of a single cluster of dwellings:

\begin{figure}[H]

{\centering \includegraphics{figures/stage2sample2} 

}

\caption{Example of a cluster of dwellings}\label{fig:sample12}
\end{figure}

or a set of clusters of dwellings:

\begin{figure}[H]

{\centering \includegraphics{figures/stage2sample3} 

}

\caption{Example of a set of clusters of dwellings}\label{fig:sample13}
\end{figure}

For communities (or parts of communities) structured in this way we use
a sampling method called the \textbf{random walk}.

\hypertarget{mapping-the-community---ribbon-communities}{%
\subsection{Mapping the community - ribbon
communities}\label{mapping-the-community---ribbon-communities}}

Ribbon communities have dwellings arranged in a line:

\begin{figure}[H]

{\centering \includegraphics{figures/stage2sample1} 

}

\caption{Dwellings arranged in a line}\label{fig:sample14}
\end{figure}

or in a several lines:

\begin{figure}[H]

{\centering \includegraphics{figures/stage2sample4} 

}

\caption{Dwellings arranged in several lines}\label{fig:sample15}
\end{figure}

For communities (or parts of communities) structured in this way we use
a sampling method called \textbf{systematic sampling}.

\hypertarget{mapping-the-community---mixed-communities}{%
\subsection{Mapping the community - mixed
communities}\label{mapping-the-community---mixed-communities}}

Some communities are a mixture of clusters and ribbons:

\begin{figure}[H]

{\centering \includegraphics{figures/stage2sample5} 

}

\caption{Mixture of clusters and ribbons}\label{fig:sample16}
\end{figure}

For mixed communities we use a mixture of the \textbf{random walk}
method (in the clusters) and \textbf{systematic sampling} (along the
ribbons).

\textbf{Segmentation} involves dividing a community into several parts
and taking part of the within-community sample from each
\textbf{segment}. With simple communities, segmentation is not required
and we take a single sample from the entire community using the
appropriate sampling method.

\hypertarget{segmentation}{%
\subsection{Segmentation}\label{segmentation}}

For more complicated communities we divide the community into several
parts or segments, such as a community made up of several clusters:

\begin{figure}[H]

{\centering \includegraphics{figures/stage2sample3} 

}

\caption{Example of a set of clusters of dwellings}\label{fig:sample17}
\end{figure}

or a community made up of several ribbons:

\begin{figure}[H]

{\centering \includegraphics{figures/stage2sample4} 

}

\caption{Dwellings arranged in several lines}\label{fig:sample18}
\end{figure}

or a mixed community:

\begin{figure}[H]

{\centering \includegraphics{figures/stage2sample5} 

}

\caption{Mixture of clusters and ribbons}\label{fig:sample19}
\end{figure}

We take a small sample from each segment using the appropriate sampling
method.

For example, with a community made up of three segments:

\begin{figure}[H]

{\centering \includegraphics{figures/stage2sample6} 

}

\caption{Community made up of three segments}\label{fig:sample20}
\end{figure}

we would take one third of the overall sample from each segment.

If the within-community sample size is twelve eligible subjects. we
would sample four eligible subjects from each segment (i.e.
\(12 / 3 = 4\)).

Dividing the sample up in this way means that we will sample from every
part of the community rather than just one part of the community.

When taking the sample we use the random walk method to take part of the
sample from clusters and the systematic sampling method to take part of
the sample from ribbons.

Segments should be either ribbons or clusters but should \textbf{never}
contain both a ribbon and a cluster. This is because clusters and
ribbons are sampled in different ways.

A dwelling can only belong to one segment. Segments should \textbf{not}
overlap.

\hypertarget{sample-dwellings}{%
\subsection{Sample dwellings}\label{sample-dwellings}}

\textbf{All} segments should be sampled.

If, for example, there are five segments in a community:

\begin{figure}[H]

{\centering \includegraphics{figures/stage2sample7} 

}

\caption{Community made up of five segments}\label{fig:sample21}
\end{figure}

and the within-community sample size is twelve eligible subjects, then
you would plan to sample two eligible subjects from each segment (i.e.
\(12 / 5 = 2.4\) \textbf{rounded down} to two) and, if necessary, return
to the \textbf{largest} segment to complete the sample.

\textbf{All} segments should be sampled, even if this means that you
take a larger sample than you expected to.

Remember that different types of segment are sampled in different ways:

\begin{itemize}
\item
  Dwellings in \textbf{cluster segments} are sampled using a method
  called the \textbf{random walk}. This involves sampling houses by
  walking in random directions within the cluster.
\item
  Dwellings in \textbf{ribbon segments} are sampled using a method
  called \textbf{systematic sampling}. This involves sampling houses at
  regular intervals along the ribbon.
\end{itemize}

We will look at each of these sampling methods in turn.

\hypertarget{random-walk-sampling}{%
\subsection{Random walk sampling}\label{random-walk-sampling}}

The \textbf{random walk} method is used to sample dwellings in
\textbf{cluster segments}. Sampling proceeds as follows:

\begin{enumerate}
\def\labelenumi{\arabic{enumi}.}
\item
  Move to the approximate centre of the cluster.
\item
  Select a \textbf{random direction} by spinning a bottle on the ground.
  The neck indicates the \textbf{sampling direction}. This is the
  direction you should walk in order to sample a dwelling. Walk in the
  sampling direction counting the dwellings that you pass. Sample the
  third \textbf{dwelling}. If there are no eligible persons in the
  selected dwelling then sample the \textbf{nearest} dwelling with an
  eligible person. Sample \textbf{all} eligible persons in the selected
  dwelling.
\item
  Apply the survey questionnaire for \textbf{all} eligible persons in
  the selected dwelling.
\item
  Select the next dwelling to sample by spinning a bottle and walking in
  the indicated direction. Count the dwellings you pass. Sample the
  \textbf{third} dwelling. If there are no eligible persons in the
  selected dwelling then sample the \textbf{nearest} dwelling with an
  eligible person. Sample all eligible persons in the selected dwelling.
  If you reach the edge of the cluster segment then return to the centre
  of the cluster and repeat step (2) above. Remember to keep count of
  the number of eligible persons sampled from the segment.
\item
  Stop sampling in the segment when you have sampled the required number
  of eligible persons from the segment. Since you sample \textbf{all}
  eligible persons in a selected dwelling, you may sample a few more
  eligible persons than expected. This is OK. Always sample \textbf{all}
  eligible persons in a selected dwelling.
\end{enumerate}

If, when you have sampled all segments, you have not sampled twelve
eligible persons, you should return to the \textbf{largest} segment to
finish sampling using the appropriate sampling method.

The random walk method is illustrated in Figure \ref{fig:sample22}.

\begin{figure}[H]

{\centering \includegraphics{figures/stage2sample8} 

}

\caption{Random walk sampling in a cluster segment}\label{fig:sample22}
\end{figure}

\hypertarget{systematic-sampling}{%
\subsection{Systematic sampling}\label{systematic-sampling}}

The \textbf{systematic sampling} method is used to sample houses in
\textbf{ribbon segments}.

Sampling proceeds as follows:

\begin{enumerate}
\def\labelenumi{\arabic{enumi}.}
\item
  Move to one end of the ribbon segment.
\item
  Walk to the other end of the segment counting the houses that you
  pass.
\item
  Calculate the \textbf{step size} by dividing the number of dwellings
  in the segment by the required sample size for the segment. Use the
  \textbf{whole number} part of the result only. Do \textbf{not} round
  up.
\item
  Pick a random number between one and the step size. This is your
  \textbf{starting point}. Select the first dwelling to sample by
  walking along the segment counting the dwellings that you pass and
  sample the dwelling indicated by the \textbf{starting point}. If there
  are no eligible persons in the selected dwelling then sample the
  \textbf{nearest} dwelling in any direction with an eligible person.
  Sample \textbf{all} eligible persons in the selected dwelling.
\item
  Select the next dwelling to sample by walking along the segment. Count
  the dwellings that you pass. Sample the dwelling indicated by the
  \textbf{step size}. If there are no eligible persons in the selected
  dwelling then sample the \textbf{nearest} dwelling in any direction
  with an eligible person. Sample \textbf{all} eligible persons in the
  selected dwelling.
\item
  Stop sampling in the segment when you reach the end of the ribbon
  segment. This may mean that you sample extra eligible persons. This is
  OK. Do \textbf{not} stop sampling from a ribbon until you reach the
  end of the ribbon.
\end{enumerate}

If, when you have sampled all segments, you have not sampled twelve
eligible persons, you should return to the \textbf{largest} segment to
finish sampling using the appropriate sampling method.

The systematic sampling method is illustrated in \ref{fig:sample23}.

\begin{figure}[H]

{\centering \includegraphics{figures/stage2sample9} 

}

\caption{Systematic sampling in a ribbon segment}\label{fig:sample23}
\end{figure}

\hypertarget{sampling-in-urban-settings}{%
\subsection{Sampling in urban
settings}\label{sampling-in-urban-settings}}

In urban areas the first stage sample is taken by replacing
sub-districts with ``sections'' and communities with city blocks.
Examples of sections may be administrative districts/sub-districts or
electoral wards.

\begin{figure}[H]

{\centering \includegraphics{figures/stage2sample10} 

}

\caption{Administrative divisions in an urban setting}\label{fig:sample24}
\end{figure}

Census enumeration areas (EAs) are usually city blocks. Central
statistics offices can usually provide lists of EAs by ``section'' and
large-scale maps of EAs selected for sampling (See \ref{fig:sample25}
and \ref{fig:sample26}). These maps make it easy to locate EAs and their
boundaries. The sample of EAs can be decided using list-based or
map-based (CSAS) sampling.

In these settings, eligible persons may be sampled by moving from
door-to-door. All dwellings in the selected block are sampled and all
eligible persons in the selected dwellings are sampled. This means that
all eligible persons in a selected block are sampled.

If city blocks are large then a type of systematic sampling may be used.
With this method a rough map of the streets in the block is made and the
number of doorways on each street is counted and copied onto the rough
street map (as shown in \ref{fig:sample27}). The total number of
doorways on all streets is calculated. A step size is calculated by
dividing the total number of doorways on all streets by the number of
dwellings to be sampled. A systematic sample along a route around the
block that includes all streets in the block is taken. Streets can be
sampled in any order. If you find that you have sampled all streets but
have not yet sampled the required number of eligible persons then you
should return to the street with the largest number of houses to collect
the remainder of the sample.

The number of blocks to be sampled will depend on the expected number of
eligible persons in each block. You should aim for an overall sample
size of about \(n = 192\). You should not sample fewer than \(m = 16\)
blocks.

\begin{figure}[H]

{\centering \includegraphics{figures/stage2sample11} 

}

\caption{Enumeration area map for a city block in Freetown, Sierra Leone}\label{fig:sample25}
\end{figure}

\begin{figure}[H]

{\centering \includegraphics{figures/stage2sample12} 

}

\caption{Enumeration area map for a city block in Addis Ababa, Ethiopia}\label{fig:sample26}
\end{figure}

\begin{figure}[H]

{\centering \includegraphics{figures/stage2sample13} 

}

\caption{Systematic sampling in a city block}\label{fig:sample27}
\end{figure}

When useful lists and maps are not available then satellite imagery
available though free services such as Google Earth
(\url{http://earth.google.com}) may be used.

The quality (resolution) of the images available from these services is
variable but is usually good enough to allow you to segment the town
into small areas of approximately equal volume (approximately the same
number of dwellings) in each:

\begin{figure}[H]

{\centering \includegraphics{figures/stage2sample14} 

}

\caption{Segmenting a town into smaller sampling areas}\label{fig:sample28}
\end{figure}

When creating segments using maps or satellite images it is a good idea
to use main roads, rivers, canals, railway lines, public parks, etc as
boundaries. This simplifies the segmentation process and also simplifies
fieldwork by making areas and their boundaries easier to locate and
sample.

The first stage sample can be list-based (such as where each area is
numbered in a systematic north to south and east to west order and a
systematic sample taken) or map-based (CSAS).

Larger scale ``maps'' of blocks to be sampled can also me made using
satellite imagery (see \ref{fig:sample29}).

\begin{figure}[H]

{\centering \includegraphics{figures/stage2sample15} 

}

\caption{A large scale “map” of a city block made from satellite imagery}\label{fig:sample29}
\end{figure}

\hypertarget{indicators}{%
\chapter{Indicators}\label{indicators}}

\hypertarget{the-ram-op-indicator-set}{%
\section{The RAM-OP indicator set}\label{the-ram-op-indicator-set}}

RAM-OP surveys collect and report on data for a broad range of
indicators relevant to older people.

These indicators cover the following dimensions:

\begin{itemize}
\tightlist
\item
  Demography and situation
\item
  Food intake
\item
  Severe food insecurity
\item
  Disability
\item
  Activities of daily living
\item
  Mental health and well-being
\item
  Dementia
\item
  Health and health-seeking behaviour
\item
  Sources of income
\item
  Water, sanitation, and hygiene
\item
  Anthropometry and screening coverage
\item
  Visual impairment
\end{itemize}

Data for a small group of miscellaneous indicators are also collected
and reported.

The RAM-OP indicator set has been designed on a modular basis. Each
module is a set of indicators relating to a single dimension from the
list given above and is collected using a dedicated set of questions and
measurements. This means that the RAM-OP questionnaire also consists of
a set of modules.

Whenever possible, RAM-OP uses standard and validated indicators and
question sets.

Indicators are described below, showing the questionnaire components
that are used to collect and record the data required, and flowcharts of
the process used to derive indicators from the collected data. Standard
symbols are used. For example:

\begin{center}\includegraphics{figures/indicators01} \end{center}

A non-standard symbol is used to show \textbf{recode operations}. A
recode operation shows changes that are made to data so that it can be
used to derive indicators without having to show many decision nodes in
the flowchart. They are also used to specify what should be done with
missing or out-of-range values. For example:

\begin{center}\includegraphics{figures/indicators02} \end{center}

\hypertarget{demography-and-situation}{%
\subsection{Demography and situation}\label{demography-and-situation}}

The demography and situation indicators are used to describe the survey
sample and are derived from this questionnaire component:

\begin{longtable}[]{@{}llll@{}}
\toprule
\begin{minipage}[t]{0.24\columnwidth}\raggedright
d1\strut
\end{minipage} & \begin{minipage}[t]{0.24\columnwidth}\raggedright
Who is answering these questions?\strut
\end{minipage} & \begin{minipage}[t]{0.24\columnwidth}\raggedright
1 = Subject

2 = Family carer

3 = Other carer

4 = Other\strut
\end{minipage} & \begin{minipage}[t]{0.24\columnwidth}\raggedright
\textbar{}\_\_\textbar{}\strut
\end{minipage}\tabularnewline
\bottomrule
\end{longtable}

\textbar{}\textbf{\textbar{} d2 How old are you (age in years)? 888 = DK
/ REFUSED \textbar{}}\textbar{}\textbf{\textbar{}}\textbar{} d3 Sex 1 =
Male; 2 = Female \textbar{}\textbf{\textbar{} d4 Marital status 1 =
Single (never married) 2 = Married 3 = Living together 4 = Divorced 5 =
Widowed 6 = Other \textbar{}}\textbar{} d5 Do you live alone? 1 = Yes; 2
= No \textbar{}\_\_\textbar{}

\hypertarget{the-ram-op-questionnaire}{%
\chapter{The RAM-OP questionnaire}\label{the-ram-op-questionnaire}}

\hypertarget{datasets}{%
\chapter{Datasets}\label{datasets}}

\hypertarget{practical}{%
\chapter{Practical Fieldwork}\label{practical}}

\hypertarget{software}{%
\chapter{RAM-OP Software}\label{software}}

\hypertarget{data-entry}{%
\section{Data entry}\label{data-entry}}

\hypertarget{data-analysis}{%
\section{Data analysis}\label{data-analysis}}

This manual covers analysing your data using the \textbf{RAnalyticFlow}
workflow. An \textbf{RAnalyticFlow} workflow may be thought of as an
\emph{``app''} that makes it easy to analyse your survey data.

To use the \textbf{RAnalyticFlow} workflow you must install:

\begin{itemize}
\item
  \textbf{The \emph{R Language for Data-Analysis and Graphics} (R)} :
  This is the \emph{``engine''} which does all the work of analysing
  your data. You can get the R installation program from:
  \url{http://cran.r-project.org}. Following are links to download
  operating sofware-specific versions of R:

  \begin{itemize}
  \item
    \href{https://cran.r-project.org/bin/linux/}{Download R for Linux}
  \item
    \href{https://cran.r-project.org/bin/macosx/}{Download R for (Mac)
    OS X}
  \item
    \href{https://cran.r-project.org/bin/windows/}{Download R for
    Windowx}
  \end{itemize}
\item
  \textbf{R packages} (libraries of functions needed to work with the
  \textbf{RAnalyticFlow} workflow) : You can install these from within
  \textbf{R} using the Package Installer function within R. The
  libraries needed are:
\end{itemize}

\begin{longtable}[]{@{}ll@{}}
\toprule
\begin{minipage}[b]{0.25\columnwidth}\raggedright
\textbf{Package}\strut
\end{minipage} & \begin{minipage}[b]{0.69\columnwidth}\raggedright
\textbf{Comments}\strut
\end{minipage}\tabularnewline
\midrule
\endhead
\begin{minipage}[t]{0.25\columnwidth}\raggedright
\textbf{rJava}\strut
\end{minipage} & \begin{minipage}[t]{0.69\columnwidth}\raggedright
Required: Used by \textbf{RAnalyticFlow}\strut
\end{minipage}\tabularnewline
\begin{minipage}[t]{0.25\columnwidth}\raggedright
\textbf{JavaGD}\strut
\end{minipage} & \begin{minipage}[t]{0.69\columnwidth}\raggedright
Required: Used by \textbf{RAnalyticFlow}\strut
\end{minipage}\tabularnewline
\begin{minipage}[t]{0.25\columnwidth}\raggedright
\textbf{codetools}\strut
\end{minipage} & \begin{minipage}[t]{0.69\columnwidth}\raggedright
Required: Used by \textbf{RAnalyticFlow}\strut
\end{minipage}\tabularnewline
\begin{minipage}[t]{0.25\columnwidth}\raggedright
\textbf{foreign}\strut
\end{minipage} & \begin{minipage}[t]{0.69\columnwidth}\raggedright
Required: Opens \textbf{EpiData} (REC) files\strut
\end{minipage}\tabularnewline
\begin{minipage}[t]{0.25\columnwidth}\raggedright
\textbf{car}\strut
\end{minipage} & \begin{minipage}[t]{0.69\columnwidth}\raggedright
Required: Used for PROBIT estimator\strut
\end{minipage}\tabularnewline
\begin{minipage}[t]{0.25\columnwidth}\raggedright
\textbf{ggplot2}\strut
\end{minipage} & \begin{minipage}[t]{0.69\columnwidth}\raggedright
Desirable: Provides many plotting functions\strut
\end{minipage}\tabularnewline
\begin{minipage}[t]{0.25\columnwidth}\raggedright
\textbf{data.table}\strut
\end{minipage} & \begin{minipage}[t]{0.69\columnwidth}\raggedright
Desirable: Speeds up working with large dataset\strut
\end{minipage}\tabularnewline
\bottomrule
\end{longtable}

The Package Installer function can be called in R using the following
command:

\begin{Shaded}
\begin{Highlighting}[]
\KeywordTok{install.packages}\NormalTok{(}\KeywordTok{c}\NormalTok{(}\StringTok{"rJava"}\NormalTok{, }\StringTok{"JavaGD"}\NormalTok{, }\StringTok{"codetools"}\NormalTok{, }
                   \StringTok{"foreign"}\NormalTok{, }\StringTok{"car"}\NormalTok{, }\StringTok{"ggplot2"}\NormalTok{, }\StringTok{"data.table"}\NormalTok{), }
                   \DataTypeTok{repos =} \StringTok{"https://cloud.r-project.org/"}\NormalTok{)}
\end{Highlighting}
\end{Shaded}

The \texttt{repos} argument in the R command above specifies the CRAN
mirror from which you to download the package/s you want to install.
Here we specify the cloud-based mirror for CRAN provided by RStudio. If
unspecified, the installiation process will prompt you to select a
mirror from which to download packages from. If you already know the URL
of the CRAN mirror you want to use, specify this in the \texttt{repos}
argument.

Note that \textbf{RAnalyticFlow} may require you to have \textbf{Java}
installed. Check the instructions on the \textbf{RAnalyticFlow}
\href{http://r.analyticflow.com/en/download/}{download page} and on this
\href{http://download.ef-prime.com/ranalyticflow/3.1.5/readme.html}{starter
guide}.

All of this software is open source and free to download, copy, and use.
It will run on Windows, Mac OS X, and Linux (and other UNIX-like)
operating systems. Your ICT department should be able to help you with
installing this software.

In addition you will also need a copy of the \textbf{RAnalyticFlow}
workflow and supporting files. These are available from:

\url{http://www.brixtonhealth.com/ramOP.rflow.zip}

You may need to extract the file from the ZIP archive before use if this
is not done automatically.

\begin{figure}[H]

{\centering \includegraphics{figures/dirStructureRAF} 

}

\caption{Directory structure of RAM-OP RAnalyticFlow package}\label{fig:raf1}
\end{figure}

Before starting to analyse your data you should create a project
directory or project folder. This is just a normal folder or directory
that can be created using your usual file manager (e.g.~Windows
ExplorerTM in WindowsTM or the FinderTM in Macintosh OS-XTM). The
project directory or project folder should contain:

\begin{enumerate}
\def\labelenumi{\arabic{enumi}.}
\item
  Your PSU file (here we assume this file is called testPSU.csv but it
  could have any name). This file must be a comma-separated-value (CSV)
  file.
\item
  Your survey data file (here we assume this file is called testSVY.rec
  but it could have any name). This file can be an EpiData (REC) file or
  a comma-separated-value (CSV) file.
\item
  The language file (always called ramOP.language.csv). This file
  provides text that is used in reports and graphics. The purpose of
  this file is to make the data analysis software produce reports in any
  language. This file must be a comma-separated-value (CSV) file.
\item
  A copy of the file ramOP.rflow.
\end{enumerate}

When you have created the project directory or project folder with the
required files you can start RAnalyticFlow.

Note: The \texttt{testSVY.rec} and \texttt{testPSU.csv} files are
example data files and are distributed with the \textbf{RAnalyticFlow}
workflow. You can use these files to practice analysing data using
\textbf{RAnalyticFlow}, and as examples of RAM-OP survey data and PSU
files.

\begin{figure}[H]

{\centering \includegraphics{figures/openProjectRAF} 

}

\caption{Creating an RAnalyticFlow project}\label{fig:raf2}
\end{figure}

Before you start work you will need to create a project for your survey:

\begin{enumerate}
\def\labelenumi{\arabic{enumi}.}
\item
  Click the \textbf{New Project\ldots{}} button
\item
  Give your project a useful (i.e.~descriptive and memorable) name. This
  might be a name that describes the survey. For example, if the survey
  was done in the Kereinik locality of West Darfur in December 2015 you
  might use the name \textbf{WD.Kereinik.Dec2015.RAMOP}
\item
  Give the location of your project directory or project folder. This is
  the directory or folder which contains your survey data file, your PSU
  file, the RAM-OP language file, and a copy of \textbf{RAMOP.rflow}
  (see previous page). The location of the project directory or project
  folder (labeled ``Path'' by the software) that \textbf{RAnalyticFlow}
  selects automatically will almost always be wrong. You need to specify
  this manually.
\item
  Click the \textbf{OK} button
\end{enumerate}

\begin{figure}[H]

{\centering \includegraphics{figures/openWorkflowRAF} 

}

\caption{Open an RAnalyticFlow workflow}\label{fig:raf3}
\end{figure}

Double click the item named \textbf{ramOP.rflow} shown in the file
manager pane of the \textbf{RAnalyticFlow} window. This will open the
data-analysis workflow which will be shown in the workflow viewer /
editor window of the \textbf{RAnalyticFlow} window.

Once you have opened the workflow you need to initialise it (i.e.~load
libraries, useful analysis function, and initialise the workspace for a
new analysis):

\begin{figure}[H]

{\centering \includegraphics{figures/runWorkflowRAF} 

}

\caption{Run an RAnalyticFlow workflow}\label{fig:raf4}
\end{figure}

Once this is done, you should:

\begin{enumerate}
\def\labelenumi{\arabic{enumi}.}
\item
  Retrieve your survey data. This can be in EpiDat (REC) format or CSV
  format. Select and run the appropriate \textbf{Survey Data} node and
  select the survey data file.
\item
  Retrieve the PSU date data. Select and run the \textbf{PSU Data} node
  and select your PSU file.
\item
  Produce the survey report and graphics. Select and run the
  \textbf{Report} node. This will take some time to complete because the
  analysis uses computer intensive techniques to make best use of the
  available data. The Report node/icon will have black lines around it
  has completed running the report.
\end{enumerate}

\hypertarget{conclusion}{%
\chapter{Conclusion}\label{conclusion}}

We live in an ageing world, where people aged 60 or over will be 2
billion or about 22\% of the world's population by 2050.

Currently, two in three people aged 60 years or older live in developing
countries. By 2050, nearly four in five older people will be living in
the developing world.

The changing demographics of ageing combined with the increasing number
of disasters will exert a disproportionate impact on the world's oldest
and poorest.

In this context, identifying the needs of older people as accurately as
possible is a necessity. More and more donors and UN agencies are now
willing to include older people in their programmes. Age markers, to
complement gender markers, will be disseminated very soon

RAM-OP is offering a fast, robust, reliable, tested and user-friendly
way of assessing the needs of older people. It can be used in
humanitarian situations as well as in development contexts. The modular
structure of RAM-OP allows for adaptations, making it exhaustive or
limited to essential indicators according to the immediate needs.

As more organisations start to use it, RAM-OP will evolve and improve.
New versions of RAM-OP can be created (for example, RAM-OP for refugee
or displaced people camps). We wish that a greater number of actors will
start using RAM-OP and make it their own.

\bibliography{book.bib}


\end{document}
